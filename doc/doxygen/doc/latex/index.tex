$\vert$ Back to the \href{https://robotology.github.io/assistive-rehab/doc/mkdocs/site/index.html}{\tt website} $\vert$

\section*{The assistive-\/rehab project}

Assistive-\/rehab is a framework for developing the assistive intelligence of \href{https://www.youtube.com/watch?v=TBphNGW6m4o}{\tt R1 robot} for clinical rehabilitation. The project is being developed within the Joint Lab between \href{https://www.iit.it}{\tt I\+IT} and \href{https://www.dongnocchi.it}{\tt Fondazionce Don Carlo Gnocchi Onlus}.

\subsection*{Library}

\href{https://robotology.github.io/assistive-rehab/doc/doxygen/doc/html/group__skeleton.html}{\tt {\bfseries {\ttfamily Assistive-\/rehab library}}} provides basic functionalities for handling skeletons. The library has definitions for\+:


\begin{DoxyItemize}
\item creating a skeleton as series of keypoints linked together with a predefined structure;
\item importing/exporting a skeleton\textquotesingle{}s structure from/into a yarp Property;
\item normalizing and scaling a skeleton;
\item optimize skeletons to deal with keypoints that cannot be observed;
\item transform skeleton\textquotesingle{}s keypoints to the desired reference system.
\end{DoxyItemize}

Additional functionalities are also included for filtering depth images and aligning two mono or multidimensional time-\/series.

\subsection*{Modules}

Assistive-\/rehab modules allow the user to\+:


\begin{DoxyItemize}
\item {\bfseries retrieve 3D skeletons}\+: given depth image from the camera and 2D skeleton data from \href{https://github.com/robotology/human-sensing}{\tt {\bfseries {\ttfamily yarp\+Open\+Pose}}}, {\bfseries {\ttfamily skeleton\+Retriever}} produces 3D skeletons and adds them in a yarp oriented database through \href{http://www.icub.org/doc/icub-main/group__objectsPropertiesCollector.html}{\tt {\bfseries {\ttfamily objects\+Properties\+Collector}}};
\item {\bfseries visualize 3D skeletons}\+: the output of {\bfseries {\ttfamily skeleton\+Retriever}} can be visualized in real-\/time on the {\bfseries {\ttfamily skeleton\+Viewer}};
\item {\bfseries analyze human motion}\+: the quality of the movement can be evaluated in real-\/time through {\bfseries {\ttfamily motion\+Analyzer}}, by specifying the tag of the metric under analysis. Metrics as the range of motion and the speed of the end-\/point are currently implemented;
\item {\bfseries recognize human actions}\+: 2D skeleton\textquotesingle{}s keypoints can feed the {\bfseries {\ttfamily action\+Recognizer}} for predicting the label of the exercise being performed;
\item {\bfseries produce a verbal feedback}\+: a feedback can be produced by {\bfseries {\ttfamily feedback\+Producer}} and translated to verbal through {\bfseries {\ttfamily feedback\+Synthetizer}};
\item {\bfseries replay and manipulate a recorded skeleton}\+: a skeleton recorded by means of {\bfseries {\ttfamily yarpdatadumper}} can be played back through {\bfseries {\ttfamily skeleton\+Player}}.
\end{DoxyItemize}

Additional details can be found in the related \href{https://robotology.github.io/assistive-rehab/doc/doxygen/doc/html/modules.html}{\tt Modules} section.

\subsection*{Applications for the robot R1}

Assistive-\/rehab applications are listed below\+:


\begin{DoxyItemize}
\item {\itshape Assistive\+Rehab.\+xml} and {\itshape Assistive\+Rehab-\/faces.\+xml}\+: for running the main demo without and with the face recognition pipeline. Tutorial for these applications can be found \href{https://robotology.github.io/assistive-rehab/doc/mkdocs/site/main_apps/}{\tt here};
\item {\itshape skeleton\+Dumper.\+xml}, {\itshape skeleton\+Dumper-\/faces.\+xml}, {\itshape Assistive\+Rehab-\/replay.\+xml}\+: for saving data without and with faces and replaying a saved experiment. Tutorial for these applications can be found \href{https://robotology.github.io/assistive-rehab/doc/mkdocs/site/replay_an_experiment/}{\tt here}.
\end{DoxyItemize}

\subsection*{Datasets}

Datasets used to train an L\+S\+TM for the action recognition pipeline can be found \href{https://github.com/robotology/assistive-rehab-storage}{\tt here}. 